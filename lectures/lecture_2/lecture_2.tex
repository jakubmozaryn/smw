\documentclass[10pt, brown]{beamer}
\usetheme{Warsaw}
\usepackage[MeX]{polski}
\usepackage[cp1250]{inputenc}  % Polskie literki...
\usepackage[polish]{babel}     %
\usepackage{parskip}
\usepackage{latexsym,gensymb,amsmath,amssymb,amsthm}
\usepackage{graphicx}
%\usepackage{verbatim}
\usepackage{ragged2e}
\usepackage{url}
%\setlength{\parindent}{0pt}
%\setlength{\parskip}{1ex}
\renewcommand{\qedsymbol}{$\square$}
\newenvironment{dowod}{{\bf Dow�d.}}{\hfill\rule[0.025cm]{0.21cm}{0.21cm}}
\newtheorem{tw}{Twierdzenie}
\newtheorem{fakt}{Fakt}
\newtheorem{lemat}{Lemat}
\newtheorem{wn}{Wniosek}

\author{dr in�. Jakub Mo�aryn, mgr in�. Jan Klimaszewski}
\institute{Instytut Automatyki i Robotyki}
\date{Warszawa, 2018}
\title{Sterowanie mechanizm�w wielocz�onowych}
\subtitle{Wyk�ad 2 - standardowe modele uk�ad�w wielocz�onowych}
\begin{document}
%
\frame{
\titlepage
}
%
\section{Manipulator RRR}
%
\frame{
    \frametitle{Model kinematyki}
}
%
\frame{
    \frametitle{Zadanie proste kinematyki}
}
%
\frame{
    \frametitle{Zadanie odwrotne kinematyki}
}
%
\frame{
    \frametitle{Zadanie proste kinematyki pr�dko�ci}
}
%
\frame{
    \frametitle{Zadanie odwrotne kinematyki pr�dko�ci}
}
%
\frame{
    \frametitle{Model dynamiki}
}
%
\section{Acrobot}
%
\frame{
    \frametitle{Model kinematyki}
}
%
\frame{
    \frametitle{Zadanie proste kinematyki}
}
%
\frame{
    \frametitle{Zadanie odwrotne kinematyki}
}
%
\frame{
    \frametitle{Zadanie proste kinematyki pr�dko�ci}
}
%
\frame{
    \frametitle{Zadanie odwrotne kinematyki pr�dko�ci}
}
%
\frame{
    \frametitle{Model dynamiki}
}
%
\section{Wahad�o odwr�cone}
%
\frame{
    \frametitle{Model kinematyki}
}
%
\frame{
    \frametitle{Zadanie proste kinematyki}
}
%
\frame{
    \frametitle{Zadanie odwrotne kinematyki}
}
%
\frame{
    \frametitle{Zadanie proste kinematyki pr�dko�ci}
}
%
\frame{
    \frametitle{Zadanie odwrotne kinematyki pr�dko�ci}
}
%
\frame{
    \frametitle{Model dynamiki}
}
%

\end{document}
