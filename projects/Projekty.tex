\documentclass[10pt,a4paper]{report}
\usepackage{polski}
\usepackage[cp1250]{inputenc}
\usepackage{amsmath}
\usepackage{amsfonts}
\usepackage{amssymb}
\usepackage{graphicx}
\author{Jakub Mo�aryn, Jan Klimaszewski}
\title{Projekty\\ \footnotesize (Sterowanie Mechanizm�w Wielocz�onowych)}
\begin{document}
	\maketitle
	\section{Tematy raport�w}
	\subsection{Temat 1}
		\subsubsection{Opis}
		\subsubsection{Za�o�enia}
		\subsubsection{Zakres}
	\newpage
	\section{Zakres projekt�w}
	\begin{enumerate}
		\item Przegl�d literatury zwi�zanej z tematem projektu,
		\item Opis metod / algorytm�w / zagadnie�,
		\item Implementacja metod / algorytm�w,
		\item Dokumentacja oprogramowania (np. Doxygen)
		\item Poster
		\item Wymogi edycyjne:
		\begin{itemize}
			\item Przygotowanie raportu w formie 10 stronicowego aryku�u sformatowanwgo zgodnie z wymaganiami IEEE \footnote{https://ieeeauthorcenter.ieee.org/create-your-ieee-article/use-authoring-tools-and-ieee-article-templates/ieee-article-templates/templates-for-transactions/}.
			\item Wykorzystanie systemu edycji tekst�w Latex.
		\end{itemize}
	\end{enumerate}
	\section{Terminy}
	\begin{enumerate}
		\item Pierwsza wersja projektu:  12.12.2018
		\item Ostateczna wersja projektu:  16.01.2018
		\item Sesja posterowa:  23.01.2018
	\end{enumerate}
\end{document}