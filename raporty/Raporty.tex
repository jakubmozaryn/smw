%& --translate-file=cp1250pl.tcx
\documentclass[10pt,a4paper]{report}
\usepackage{polski}
\usepackage[latin1]{inputenc}
\usepackage{amsmath}
\usepackage{amsfonts}
\usepackage{amssymb}
\usepackage{graphicx}
\author{Jakub Mo?aryn, Jan Klimaszewski}
\title{Raporty z przedmiotu Sterowanie Mechanizm�w Wielocz?onowych}
\begin{document}
	\maketitle
	\section{Tematy raport�w}
\begin{enumerate}
	\item Notacja DH i jej modyfikacje. (JK)
	\item Opis mechanizm�w wielocz?onowych z elementami elastycznymi.(JK)
	\item Zadanie odwrotne kinematyki - przegl?d metod. (JK)
	\item Zastosowanie kwaternion�w do opisu mechanizm�w wielocz?onowych. (JM)
	\item Opis zamknietych mechanizm�w wielocz?onowych. (JK)
	\item Opis drzewiastych mechanizm�w wielocz?onowych. (JK)
	\item Opis modelu robota Puma 560 - parametry modelu, model kinematyki i dynamiki. (JM)
	\item Opis modelu robota Universal UR5 - parametry modelu, model kinematyki i dynamiki. (JM)
	\item Algorytm Walkera-Orina wyznaczania dynamiki MW. (JM)
	\item Linearyzacja modelu robota w wybranym punkcie pracy. (JM)
	\item Opis przegub�w kulistych w MW. (JM)
	\item Algorytm Fatherstone'a wyznaczania dynamiki MW. (JM)
	\item Instalacja i uruchomienie symulacji mechanizmu wielocz?onowego w srodowisku V-Rep, komunikacja srodowisk V-Rep i Matlab. (JM)
	\item Opis modelu robota Darwin-OP i symulacja w srodowisku ROS/Gazebo. (JM)
	\item Uklady niedosterowane. (JK)
	\item Sposoby opisu kolizji. (JK)
	\item Sposoby amortyzacji kolizji, np. stopa mechanizmu kroczacego.
	\item Kinematyczne r�wnania wiez�w dla po?o?enia, pr?dko?ci i przyspieszenia. Holonomiczno??. (JK)
	\item Jakobian analityczny vs geometryczny - r�?ne sposoby zapisu orientacji i obrotu we wsp�?rz?dnych zewn?trznych. (JK)
	\item Identyfikacja parametr�w kinematycznych i dynamicznych robot�w.(JM)
	\item Model dwuno?nej maszyny krocz?cej. (JK)
	\item Strategie poruszania si? dwuno?nych maszyn krocz?cych - metoda ZMP. (JK)
	\item Model czterono?nej maszyny krocz?cej. (JK)
    \item Strategie poruszania si? czterono?nych maszyn krocz?cych. (JK)
	\item R�wnowaga i stabilno?? w odniesieniu do maszyn krocz?cych. (JK)
	\item Model motocyklu/roweru w uj?ciu mechanizm�w wielocz?onowych. (JK)
	\item Heurystyki dla planowania trasy. (JK)
	\item Sposoby dyskretyzacji przestrzeni przeszukiwania. (JK)
	\item Sposoby redukcji przestrzeni przeszukiwania. (JK)
	\item Przestrze? przeszukiwania z uwzgl?dnieniem czasu - omijanie przeszk�d w ruchu. (JK)
	\item Algorytm Dijkstry i A*. (JK)
	\item Algorytm D*, D* Light oraz D* Extra Lite. (JK)
\end{enumerate}
	\section{Zakres raport�w}
	\begin{enumerate}
		\item Przegl?d literatury zwi?zanej z tematem raportu.
		\item Opis metody / algorytmu / zagadnienia.
		\item W przypadku zagadnie? teoretycznych, zastosowanie metody/algorytmu na prostym przyk?adzie (nie trzeba kodowa?, mo?na ewentualnie skorzysta? z Matlab Symbolic Toolbox).
		\item W przypadku rozwi?z? praktycznych (uruchomienie symulacji), opis w postaci tutorialu.
		\item Wymogi edycyjne:
		\begin{itemize}
			\item Strona tytu?owa, 7-10 stron tekstu + 1-2 strony literatury.
			\item Wykorzystanie systemu edycji tekst�w Latex, szablon \textbf{report}.
			\item Uk?ad: A4, czcionka: Times New Roman, 10pt.
		\end{itemize}
	\end{enumerate}
	\section{Terminy}
	\begin{enumerate}
		\item Pierwsza wersja raportu:  24.10.2018
		\item Ostateczna wersja raportu:  7.11.2018
	\end{enumerate}
\end{document}